% !TeX program = pdfLaTeX
\documentclass[12pt]{article}
\usepackage{amsmath}
\usepackage{graphicx,psfrag,epsf}
\usepackage{enumerate}
\usepackage{natbib}
\usepackage{textcomp}
\usepackage[hyphens]{url} % not crucial - just used below for the URL
\usepackage{hyperref}

%\pdfminorversion=4
% NOTE: To produce blinded version, replace "0" with "1" below.
\newcommand{\blind}{0}

% DON'T change margins - should be 1 inch all around.
\addtolength{\oddsidemargin}{-.5in}%
\addtolength{\evensidemargin}{-.5in}%
\addtolength{\textwidth}{1in}%
\addtolength{\textheight}{1.3in}%
\addtolength{\topmargin}{-.8in}%

%% load any required packages here



% tightlist command for lists without linebreak
\providecommand{\tightlist}{%
  \setlength{\itemsep}{0pt}\setlength{\parskip}{0pt}}



\usepackage{float}
\usepackage{mathtools}
\usepackage{natbib}
\usepackage[linesnumbered,ruled,vlined]{algorithm2e}
\usepackage{verbatim}
\usepackage{amsthm}
\usepackage{comment}
\usepackage{amsfonts}

\begin{document}


\def\spacingset#1{\renewcommand{\baselinestretch}%
{#1}\small\normalsize} \spacingset{1}


%%%%%%%%%%%%%%%%%%%%%%%%%%%%%%%%%%%%%%%%%%%%%%%%%%%%%%%%%%%%%%%%%%%%%%%%%%%%%%

\if0\blind
{
  \title{\bf Manifold Clustering in the Generalized Random Dot Product
Graph}

  \author{
        John Koo \\
    Department of YYY, University of XXX\\
      }
  \maketitle
} \fi

\if1\blind
{
  \bigskip
  \bigskip
  \bigskip
  \begin{center}
    {\LARGE\bf Manifold Clustering in the Generalized Random Dot Product
Graph}
  \end{center}
  \medskip
} \fi

\bigskip
\begin{abstract}
The text of your abstract. 200 or fewer words.
\end{abstract}

\noindent%
{\it Keywords:} block models, community detection, coordinate descent,
latent structure models, manifold clustering, random dot product graph
\vfill

\newpage
\spacingset{1.45} % DON'T change the spacing!

\newcommand{\diag}{\mathrm{diag}}
\newcommand{\tr}{\mathrm{Tr}}
\newcommand{\blockdiag}{\mathrm{blockdiag}}
\newcommand{\indep}{\stackrel{\mathrm{ind}}{\sim}}
\newcommand{\iid}{\stackrel{\mathrm{iid}}{\sim}}
\newcommand{\Bernoulli}{\mathrm{Bernoulli}}
\newcommand{\Betadist}{\mathrm{Beta}}
\newcommand{\BG}{\mathrm{BernoulliGraph}}
\newcommand{\Uniform}{\mathrm{Uniform}}
\newcommand{\PABM}{\mathrm{PABM}}
\newcommand{\RDPG}{\mathrm{RDPG}}
\newcommand{\GRDPG}{\mathrm{GRDPG}}
\newcommand{\Multinomial}{\mathrm{Multinomial}}
\newtheorem{theorem}{Theorem}
\newtheorem{lemma}{Lemma}
\newtheorem{proposition}{Proposition}
\theoremstyle{remark}
\newtheorem{remark}{Remark}
\theoremstyle{definition}
\newtheorem{definition}{Definition}
\newtheorem{example}{Example}
\newcommand{\dd}{\mathrm{d}}
\newcommand{\as}{\stackrel{\mathrm{a.s.}}{\to}}
\newcommand{\ER}{\text{Erd\"{o}s-R\'{e}nyi}}

\hypertarget{introduction}{%
\section{Introduction}\label{introduction}}

We define a \emph{Bernoulli graph} as a random graph model for which
edge probabilities are contained in an edge probability matrix
\(P \in [0, 1]^{n \times n}\), and an edge occurs between vertices \(i\)
and \(j\) with probability \(P_{ij}\). Common random graph models then
impose structure on \(P\), based on various assumptions about the way in
which the data are generated, or to allow \(P\) to be estimated. One
example is the \(\text{Erd\"{o}s-R\'{e}nyi}\) model, in which all edge
probabilities are fixed, i.e., \(P_{ij} = p\).

One common analysis task for graph and network data is community
detection, which assumes that each vertex of a graph has a hidden
community label. The goal of the analysis is then to estimate these
labels upon observing a graph. In order to perform this analysis as a
statistical inference task is to define a probability model with
inherent community structure. We call such models \emph{block models}:
First, each vertex is assigned a label
\(z_1, ..., z_n \in \{1, 2, ..., K\}\) where \(K \ll n\). Then each edge
probability \(P_{ij}\) is said to depend on the labels \(z_i\) and
\(z_j\), possibly along with some other parameters. For example, the
stochastic block model (SBM) sets a fixed edge probability for each pair
of communities, i.e., \(P_{ij} = \omega_{z_i, z_j}\). The
degree-corrected block model (DCBM) assigns an additional parameter
\(\theta_i\) to each vertex by which edge probabilities are scaled,
i.e., \(P_{ij} = \theta_i \theta_j \omega_{z_i, z_j}\). The popularity
adjusted block model (PABM) assigns \(K\) parameters to each vertex
\(\lambda_{i1}, \lambda_{i2}, ..., \lambda_{iK}\) that describe that
vertex's affinity toward each community; the edge probability between
vertices \(i\) and \(j\) is then defined as the product of vertex
\(i\)'s affinity toward vertex \(j\)'s community and vertex \(j\)'s
affinity toward vertex \(i\)'s community, i.e.,
\(P_{ij} = \lambda_{i z_j} \lambda_{j z_i}\).

The three block model types, as well as the
\(\text{Erd\"{o}s-R\'{e}nyi}\) model, impose structure on \(P\),
including on the rank of \(P\). \(P\) has rank 1 for the
\(\text{Erd\"{o}s-R\'{e}nyi}\) model, rank \(K\) for the SBM and DCBM,
and rank \(K^2\) for the PABM. This provides the intuition behind
another family of Bernoulli graphs called the \emph{random dot product
graph} (RDPG) and \emph{generalized random dot product graph} (GRDPG).
In the RDPG, each vertex has a corresponding latent vector in
\(d\)-dimensional Euclidean space, where \(d\) is the rank of \(P\) and
\(P\) is positive semidefinite. Then the edge probability between each
pair of vertices is defined as the inner product between the
corresponding latent vectors, i.e., \(P_{ij} = x_i^\top x_j\). If the
latent vectors are collected in a data matrix
\(X = \bigl[ x_1 \mid \cdots \mid x_n \bigr]^\top\), then the edge
probability matrix for the RDPG is \(P = X X^\top\). Similarly, the edge
probability between each pair of vertices for the GRDPG is defined as
the indefinite inner product between the corresponding latent vectors,
i.e., \(P_{ij} = x_i^\top I_{p,q} x_j\), where
\(I_{p,q} = \mathrm{blockdiag}(I_p, -I_q)\) and \(p + q = d\). Then the
edge probability matrix for the GRDPG is \(P = X I_{p,q} X^\top\). This
allows for a model similar to the RDPG for non-positive semidefinite
\(P\). While the RDPG and GRDPG do not necessarily have community
structure, it has been shown that block models are specific cases of the
RDPG or GRDPG in which latent vectors are organized by community. This
includes the SBM, in which communities correspond to point masses, DCBM,
in which communities correspond to line segments, and PABM, in which
communities correspond to orthogonal subspaces. In this work, we extend
this idea to communities organized into more general latent structures.
In particular, we assume that each community corresponds to a manifold
in the latent space.

\hypertarget{generalized-random-dot-product-graphs-with-community-structure}{%
\section{Generalized Random Dot Product Graphs with Community
Structure}\label{generalized-random-dot-product-graphs-with-community-structure}}

All Bernoulli graphs are generalized random dot product graphs. Whether
this is useful for inference depends on the structure of the latent
space. In the case of the \(\text{Erd\"{o}s-R\'{e}nyi}\) model, SBM,
DCBM, and PABM, the latent structure is linear, and the linearity can be
exploited for community detection and parameter estimation. In this
section, we discuss general, often nonlinear latent structure models,
focusing on those with community structure.

To motivate this, consider a generalization of the
\(\text{Erd\"{o}s-R\'{e}nyi}\) model. Recall that when viewed as an
RDPG, the latent space of an \(\text{Erd\"{o}s-R\'{e}nyi}\) model
consists of one point in Euclidean space. In the following example,
instead of fixing the edge probability, it is sampled from a
distribution in such a way that when viewed as an RDPG, the latent space
consists of a curve.

\begin{example}[Hierarchical $\text{Erd\"{o}s-R\'{e}nyi}$ model]
In the $\text{Erd\"{o}s-R\'{e}nyi}$ model, the edge probability matrix has a fixed value $[P_{ij}] \equiv p \in [0, 1]$. 

Suppose that we have a random dot product graph in which the latent space is $\mathbb{R}^2$ and latent vectors are drawn uniformly from the quarter circle defined by $g(t) = \begin{bmatrix} \cos(\frac{\pi}{2} t) & \sin(\frac{\pi}{2} t) \end{bmatrix}^\top$, $0 \leq t \leq 1$. 
Then it can be shown that in this model, instead of a fixed $P_{ij} = p$, the edge probabilities are distributed with density $f(p) = \frac{2}{\pi - 2} \Big(\frac{1}{\sqrt{1 - p^2}} - 1 \Big)$. 
\end{example}

By changing the latent structure from a point mass to a curve, we are
able to come up with more flexible Bernoulli graph models in which edge
probabilities follow more general probability distributions. Community
structure then can be added by sampling latent vectors from multiple
curves. Then the adjacency spectral embedding of the resulting graph
allows us to recover that community structure. This is illustrated in
the following example.

\begin{example}
Define two one-dimensional manifolds in $\mathbb{R}^2$ by $f_1(t) = \begin{bmatrix} \cos(\frac{\pi}{3} t) & \sin(\frac{\pi}{3} t) \end{bmatrix}^\top$ and $f_2(t) = \begin{bmatrix} 1 - \cos(\frac{\pi}{3} t) & 1 - \sin(\frac{\pi}{3} t) \end{bmatrix}^\top$.
Draw $t_1, ..., t_n \stackrel{\mathrm{iid}}{\sim}\mathrm{Uniform}(0, 1)$ and $z_1, ..., z_n \stackrel{\mathrm{iid}}{\sim}\mathrm{Multinomial}(\frac{1}{2}, \frac{1}{2})$, and compute latent vectors $x_i = f_{z_i}(t_i)$, which are collected in data matrix $X = \begin{bmatrix} x_1 & \cdots & x_n \end{bmatrix}^\top$. 
Finally, let $A \sim \mathrm{RDPG}(X)$. Fig. \ref{fig:example1} shows the latent configuration drawn from this latent distribution, a random dot product graph drawn from the latent configuration, and the adjacency spectral embedding of the graph. 
Although the community structure is not obvious from the graph, the embedding shows a clear separation between the two communities. 

\begin{figure}[H]

{\centering \includegraphics{draft_files/figure-latex/example1-1} 

}

\caption{Manifold block model described in Example 1. The latent configuration is on the left, a random dot product graph drawn from the latent configuration is on the middle, and the ASE is on the right.}\label{fig:example1}
\end{figure}
\end{example}

We now formally define the manifold block model.

\begin{definition}[Manifold block model]
\label{def:manifold-block-model}
Let $p, q \geq 0$, $d = p + q \geq 1$, $1 \leq r < d$, $K \geq 2$, and $n \geq 1$ be integers.
Define manifolds $\mathcal{M}_1, ..., \mathcal{M}_K \subset \mathcal{X}$ for $\mathcal{X} = \{x, y \in \mathbb{R}^d : x^\top I_{p,q} y \in [0, 1] \}$ each by continuous function $g_k : [0, 1] \to \mathcal{X}$, 
and probability distributions $F_1, ..., F_K$ each with support $[0, 1]^r$. 
Then the following mixture model is a manifold block model: 

\begin{enumerate}
  \item Draw labels $z_1, ..., z_n \stackrel{\mathrm{iid}}{\sim}\mathrm{Multinomial}(\alpha_1, ..., \alpha_K)$. 
  \item Draw latent vectors by first drawing each $t_i \stackrel{\mathrm{ind}}{\sim}F_{z_i}$ and then compute each $x_i = g_{z_i}(t_i)$. 
  \item Let $X = \begin{bmatrix} x_1 & \cdots & x_n \end{bmatrix}^\top$, and draw $A \sim \mathrm{RDPG}(X; \rho_n)$ or $A \sim \mathrm{GRDPG}_{p,q}(X; \rho_n)$. 
\end{enumerate}
\end{definition}

\hypertarget{methods}{%
\section{Methods}\label{methods}}

We provide two approaches to community detection for the manifold block
model. First, we consider the case in which communities correspond to
manifolds in the latent space that do not intersect and are separated by
some finite distance. In this scenario, we use the convergence of the
ASE to show that single linkage clustering on the latent space produces
a clustering such that the total number of misclustered vertices goes to
zero, with high probability.

Next, we consider the case in which communities correspond to
one-dimensional manifolds in the latent space and may or may not
intersect. In this scenario, we propose an alternating coordinate
descent algorithm that alternates between estimating the structure of
the manifolds and the community labels, which we call \(K\)-curves
clustering. We again use the convergence of the ASE to show that under
certain conditions, \(K\)-curves clustering produces a clustering such
that the proportion of misclustered vertices goes to zero, with high
probability.

\hypertarget{nonintersecting-manifolds}{%
\subsection{Nonintersecting Manifolds}\label{nonintersecting-manifolds}}

\label{section:nonintersecting}

In this section, we consider the following scenario: Suppose that each
community is represented by a closed manifold \(\mathcal{M}_k\),
\(k \in \{1, ..., K\}\) in the latent space of a RDPG or GDRPG. Define
\(\delta = \min\limits_{k \neq \ell} \min\limits_{x \in \mathcal{M}_k, y \in \mathcal{M}_\ell} \|x - y\|\),
the minimum distance between two manifolds. We assume that
\(\delta > 0\), i.e., the manifolds do not intersect.

In the noiseless setting, if the subsample on each manifold is
sufficiently dense, it is possible to construct for each manifold an
\(\eta_k\)-neighborhood graph for each manifold for some \(\eta_k > 0\)
such that the graph is connected. Then if
\(\max_k \eta_k = \eta < \delta\), an \(\eta\)-neighborhood graph for
the entire sample will consist of \(K\) disconnected subgraphs that map
onto each manifold. Equivalently, we can apply single-linkage
clustering. The remainder of this section explores under which
conditions these criteria are met for the latent configuration, in which
latent vectors lie exactly on manifolds, as well as the ASE, which
introduces noise.

\begin{algorithm}[h]
\DontPrintSemicolon
\SetAlgoLined
\KwData{Adjacency matrix $A$, number of communities $K$, embedding dimensions $p$ and $q$.}
\KwResult{Community assignments $z_1, ..., z_n \in \{1, ..., K\}$.}
Compute $\hat{X}$, the ASE of $A$ using the $p$ most positive and $q$ most negative eigenvalues and their corresponding eigenvectors.\;
Apply single linkage clustering with $K$ communities on $\hat{X}$.\;
\caption{ASE clustering for nonintersecting communities.}
\end{algorithm}

Let \(F_k\) be a probability distribution with support
\(\mathcal{M}_k\). Then we define a mixture model as follows:

\begin{enumerate}
\def\labelenumi{\arabic{enumi}.}
\tightlist
\item
  Draw labels
  \(z_1, ..., z_n \stackrel{\mathrm{iid}}{\sim}\mathrm{Multinomial}(\alpha_1, ..., \alpha_K)\).
\item
  Draw latent vectors each as
  \(x_i \stackrel{\mathrm{ind}}{\sim}F_{z_i}\) for distributions
  \(F_1, ..., F_K\) with respective supports
  \(\mathcal{M}_1, ..., \mathcal{M}_K\).
\item
  Let \(X = \begin{bmatrix} x_1 & \cdots & x_n \end{bmatrix}^\top\), and
  draw \(A \sim \mathrm{RDPG}(X; \rho_n)\) or
  \(A \sim \mathrm{GRDPG}_{p,q}(X; \rho_n)\).
\end{enumerate}

Note that here, we redefine the model to ignore \(g_1, ..., g_K\), the
parameterizations of each manifold. Instead, we sample points directly
on the manifolds themselves. We will return to the parameterizations in
Section \ref{section:intersecting}.

\begin{theorem}[Community detection for the GRDPG for which the communities come from nonintersecting manifolds]
\label{theorem:nonintersect-grdpg}
Let $x_1, ..., x_n$ be points sampled on $K$ compact, connected manifolds $\mathcal{M}_1, ..., \mathcal{M}_K \subset \mathbb{R}^d$ each with probability measures $F_1, ..., F_K$, and the manifolds are separated by distance at least $\delta = \min\limits_{k \neq \ell} \min\limits_{x_i \in \mathcal{M}_k, x_j \in \mathcal{M}_\ell} \| x_i - x_j \| > 0$. 
Let $X = \begin{bmatrix} x_1 & \cdots & x_n \end{bmatrix}^\top$ and $A \sim \mathrm{GRDPG}_{p,q}(X; \rho_n)$ for some $p, q \in \mathbb{N}_0$ such that $p + q = d$ and sparsity parameter $\rho_n$ that satisfies $n \rho_n = \omega(\log^c n)$ for some $c > 1$. 
Define $A_n(\eta)$ as the event that an $\eta$-neighborhood graph constructed from the ASE of $A$ consists of exactly $K$ disconnected subgraphs that map exactly to each manifold. 
Then for any $\eta \in (0, \delta)$, $\lim\limits_{n \to \infty} P(A_n(\eta)) = 1$. 
\end{theorem}

If the manifolds are one-dimensional, then a more precise rate of
convergence can be derived.

\hypertarget{intersecting-manifolds}{%
\subsection{Intersecting Manifolds}\label{intersecting-manifolds}}

\label{section:intersecting}

In this section, we again consider the setting for the RDPG or GRDPG in
which each community lies on a manifold in the latent space. However,
this time, we do not assume that the manifolds are nonintersecting. We
also restrict this case to one-dimensional manifolds which are each
described by \(g_k : [0, 1] \to \mathcal{X}\). Then we define a mixture
model as follows:

\begin{enumerate}
\def\labelenumi{\arabic{enumi}.}
\tightlist
\item
  Draw \(t_1, ..., t_n \stackrel{\mathrm{iid}}{\sim}F\) for probability
  distribution \(F\) with support \([0, 1]\).
\item
  Draw
  \(z_1, ..., z_n \stackrel{\mathrm{iid}}{\sim}\mathrm{Multinomial}(\alpha_1, ..., \alpha_K)\),
  the community labels.
\item
  Let each \(x_i = g_{z_i}(t_i)\) be the latent vector for vertex
  \(v_i\), and collect the latent vectors into matrix
  \(X = \begin{bmatrix} x_1 & \cdots & x_n \end{bmatrix}^\top\).
\item
  Draw \(A \sim \mathrm{RDPG}(X)\) or
  \(A \sim \mathrm{GRDPG}_{p,q}(X)\).
\end{enumerate}

\begin{algorithm}[h]
\DontPrintSemicolon
\SetAlgoLined
\KwData{Adjacency matrix $A$, number of communities $K$, embedding dimensions $p$, $q$, stopping criterion $\epsilon$}
\KwResult{Community assignments $1, ..., K$, curves $g_1, ..., g_K$}
Compute $X$, the ASE of $A$ using the $p$ most positive and $q$ most negative eigenvalues and their corresponding eigenvectors.\;
Initialize community labels $z_1, ..., z_n$.\;
\Repeat {the change in $\sum_k \sum_{i \in C_k} \|x_i - g_k(t_i)\|^2$ is less than $\epsilon$} {
\For {$k = 1, ..., K$} {
Define $X_k$ as the rows of $X$ for which $z_i = k$.\;
Fit curve $g_k$ and positions $t_{k_i}$ to $X_k$ by minimizing $\sum_{k_i} \|x_{k_i} - g_k(t_{k_i})\|^2$.\;
}
\For {$k = 1, ..., K$} {
Assign $z_i \leftarrow \arg\min_\ell \|x_i - g_\ell(t_i)\|^2$.\
}
}
\caption{$K$-curves clustering.}
\end{algorithm}

rves clustering.\} \textbackslash end\{algorithm\}

\(K\)-curves clustering assumes that the functional form of \(g_k\) is
known. The choice of \(g_k\) affects the difficulty of the algorithm. As
a balance between flexibility and ease of estimation, we consider the
case where each \(g_k\) is a Bezier polynomial of degree \(R\) with
coefficients \(p_k\). Then we have
\(g_k(t) = g(t; p_k) = \sum_{r=0}^R p_k^{(r)} \binom{R}{r} (1-t)^{R-r} t^r\).

Given \(\{t_i\}\) and \(\{z_i\}\), it is straightforward to obtain
\(\hat{p}_k = \arg\min_p \sum_{k_i} \|x_{k_i} - g_k (t_{k_i}; p)\|^2\)
\[\hat{p}_k = (T_k^\top T_k)^{-1} T_k^\top X_k,\] where \(T_k\) is an
\(n_k \times (R+1)\) matrix with rows
\(\begin{bmatrix} (1 - t_{k_i})^R & (1 - t_{k_i})^{R-1} t_{k_i} & \cdots & (1 - t_{k_i}) t_{k_i}^{R-1} & t_{k_i}^R \end{bmatrix}\).
Estimation of \(\{t_i\}\) given \(\{z_i\}\) and \(\{p_k\}\) is more
difficult. Each \(t_i\) can be estimated separately:
\begin{equation} \label{eq:min-t}
\hat{t}_i = \arg\min_t \|x_i - g(t; p_{z_i})\|^2. 
\end{equation} This is equivalent to solving
\(0 = (x_i - g(t; p_{z_i}))^\top (\dot{g}(t; p_{z_i}))\). Setting
\(c^{(s)} = \sum_{r=0}^s (-1)^{s-r} \binom{R}{r} p^{(r)}_{z_i}\) for
\(s \neq 0\) and \(c^{(0)} = p^{(0)}_{z_i} - x_i\), let
\(c = \begin{bmatrix} c^{(0)} & \cdots & c^{(R)} \end{bmatrix}^\top\).
Then solving Eq. \ref{eq:min-t} is equivalent to finding the real roots
of a polynomial with coefficients that are the sums of the reverse
diagonals of \(C D^\top\), where \(C_{ij} = c_{ij} (-1)^i \binom{R}{i}\)
and \(D_{ij} = c_{i-1,j} (-1)^{i-1} \binom{R-1}{i-1}\).

\begin{algorithm}[h]
\DontPrintSemicolon
\SetAlgoLined
\KwData{Adjacency matrix $A$, number of communities $K$, embedding dimensions $p$, $q$, stopping criterion $\epsilon$, $m_k \leq n_k$ known community assignments for each community}
\KwResult{Community assignments $1, ..., K$, curves $g_1, ..., g_K$}
Compute $X$, the ASE of $A$ using the $p$ most positive and $q$ most negative eigenvalues and their corresponding eigenvectors.\;
Fit curves $g_1, ..., g_K$ using each of the $m_1, ..., m_K$ points with known community labels by minimizing $\sum_{j=1}^{m_i} \|x_j - g_k(t_j)\|^2$.\;
Assign labels $z_1, ..., z_n$ to each $x_1, ..., x_n$ by minimizing $\|x_i - g_k(t_i)\|^2$ for $k$, holding the initial known labels constant.\; 
\Repeat {the change in $\sum_k \sum_{i \in C_k} \|x_i - g_k(t_i)\|^2$ is less than $\epsilon$} {
\For {$k = 1, ..., K$} {
Define $X_k$ as the rows of $X$ for which $z_i = k$.\;
Fit curve $g_k$ and positions $t_{k_i}$ to $X_k$ by minimizing $\sum_{k_i} \|x_{k_i} - g_k(t_{k_i})\|^2$.\;
}
\For {$k = 1, ..., K$} {
Assign $z_i \leftarrow \arg\min_\ell \|x_i - g_\ell(t_i)\|^2$, holding the known initial labels constant.\
}
}
\caption{Semi-supervised $K$-curves clustering.}
\end{algorithm}

upervised \(K\)-curves clustering.\} \textbackslash end\{algorithm\}

\begin{theorem}
Let each $g(\cdot; p_k)$ be a nonintersecting Bezier polynomial of order $R$, 
and a GRDPG is drawn from vectors that lie on the curves. 
Suppose we observe the true labels of $m_k$ vertices from each community, and each $m_k > R + 1$. Suppose further that latent vectors $x_j = g(t_i; p_{z_j})$ that correspond to vertices with observed labels are such that 
Then as $n \to \infty$, the proportion of misclustered vertices from $K$-curves clustering approaches $0$ with probability $1$.
\end{theorem}

\hypertarget{examples}{%
\section{Examples}\label{examples}}

\begin{example}
Here, $K = 2$ with $g_1(t) = \begin{bmatrix} t^2 & 2 t (1 - t) \end{bmatrix}^\top$ and $g_2(t) = \begin{bmatrix} 2 t (1 - t) & (1 - t) ^ 2 \end{bmatrix}^\top$. We draw $n_1 = n_2 = 2^8$ points uniformly from each curve. 

\begin{figure}[H]

{\centering \includegraphics{draft_files/figure-latex/unnamed-chunk-2-1} 

}

\caption{Latent positions, labeled by curve/community.}\label{fig:unnamed-chunk-2}
\end{figure}

We draw $A \sim \mathrm{RDPG}(X)$ and obtain the following ASE:

\begin{figure}[H]

{\centering \includegraphics{draft_files/figure-latex/unnamed-chunk-3-1} 

}

\caption{ASE of an RDPG drawn from the latent positions, labeled by curve/community.}\label{fig:unnamed-chunk-3}
\end{figure}

We then try applying $K$-curves clustering to this graph. 
The first three are with random initial labels, forcing the intercept to be zero. 
The fourth initializes the labels randomly but allows the intercept to be nonzero. 
The fifth initializes the labels by spectral clustering with the normalized Laplacian, again forcing the intercept to be zero. 
The sixth also initializes via spectral clustering but allows the intercept to be nonzero. 





\begin{figure}[H]

{\centering \includegraphics{draft_files/figure-latex/unnamed-chunk-6-1} 

}

\caption{Clustering loss vs. iteration for each run of K-curve clustering.}\label{fig:unnamed-chunk-6}
\end{figure}

\begin{figure}[H]

{\centering \includegraphics{draft_files/figure-latex/unnamed-chunk-7-1} 

}

\caption{ASE labeled by estimated community labels for each initialization strategy.}\label{fig:unnamed-chunk-7}
\end{figure}

\end{example}

\begin{example}[Macaque visuotactile brain areas and connections \citep{https://doi.org/10.1111/j.1460-9568.2006.04678.x}]


\begin{center}\includegraphics{draft_files/figure-latex/unnamed-chunk-8-1} \end{center}




\begin{center}\includegraphics{draft_files/figure-latex/unnamed-chunk-10-1} \end{center}


\begin{center}\includegraphics{draft_files/figure-latex/unnamed-chunk-11-1} \end{center}

\end{example}

\begin{example}[Non-intersecting curves]


\begin{center}\includegraphics{draft_files/figure-latex/unnamed-chunk-12-1} \end{center}






\begin{center}\includegraphics{draft_files/figure-latex/unnamed-chunk-15-1} \end{center}

\end{example}

\hypertarget{simulation-study}{%
\section{Simulation Study}\label{simulation-study}}

\begin{center}\includegraphics{draft_files/figure-latex/unnamed-chunk-17-1} \end{center}

\hypertarget{discussion}{%
\section{Discussion}\label{discussion}}

\appendix

\section{Proofs of Theorems}

\begin{lemma}
\label{lemma:no-noise}
Let $x_1, ..., x_n$ be drawn from $K$ compact, connected manifolds $\mathcal{M}_1$, ..., $\mathcal{M}_K$ each with probability measures $F_1$, ..., $F_K$, and the manifolds are separated by distance at least $\delta > 0$. 
Suppose that for any $\epsilon > 0$ and $x$ drawn from each $F_k$, on $\mathcal{M}_k$, $F(B(x, \epsilon)) > 0$ where $B(x, \epsilon)$ is the open ball of radius $\epsilon$ centered at $x$. 
Let $E_n(\eta)$ denote the event that an $\eta$-neighborhood graph constructed from $x_1, ..., x_n$ is comprised of exactly $K$ disjoint subgraphs that map to each of the $K$ manifolds. 
Then if each $n_k \to \infty$ as $n \to \infty$, $\lim\limits_{n \to \infty} P(E_n(\eta)) = 1$ for each $\eta \in (0, \delta)$. 
\end{lemma}

\begin{proof}
It is clear that if $\eta \in (0, \delta)$, an $\eta$-neighborhood graph constructed from the sample will always consist of at least $K$ disjoint subgraphs for which no subgraph contains vertices belonging to points from two different manifolds. 
Then it is sufficient to show that for a sufficiently large $n$, any $\eta$-neighborhood graph (where $\eta \in (0, \delta)$) will achieve $E_n$. 

Define each $E_{n_k}^{(k)}(\eta)$ as the event that if a sub-sample of size $n_k$ drawn from manifold $\mathcal{M}_k$, every $x \in \mathcal{M}_k$ is within distance $\eta$ of some $x_j$ of the sub-sample. 
Then if $E_{n_k}^{(k)}(\eta)$ is true, the $\eta$-neighborhood graph results in a connected subgraph for points within the $k^{th}$ manifold. 
By lemma 2 of \citet{trosset2020rehabilitating}, $P((E_{n_k}^{(k)}(\eta))^c) \leq \ell_k (1 - b_k)^{n_k}$ for some $\ell_k \in \mathbb{N}$ and $b_k \in (0, 1]$. 
If each $E_{n_k}^{(k)}(\eta)$ is true, then $E_n$ is achieved, so $E_n(\eta) = \bigcap_k E_{n_k}^{(k)}(\eta))$. 
$\bigcap_k E_{n_k}^{(k)}(\eta) = \Big( \bigcup_k (E_{n_k}^{(k)}(\eta))^c \Big)^c$, so it is sufficient to show $\lim\limits_{n \to \infty} P \big( \bigcup_k (E_{n_k}^{(k)}(\eta))^c \big) \to 0$. 

$$
\begin{aligned}
P \big( \bigcup_k(E_{n_k}^{(k)})^c \big) & \leq \sum_k P \big((E_{n_k}^{(k)})^c \big) \\
& \leq \sum_k \ell_k (1 - b_k)^{n_k} \\
& \leq K \ell_{\max} (1 - b_{\min})^{n_{\min}},
\end{aligned}
$$

which tends to $0$ as $n \to \infty$. 
\end{proof}

\begin{proof}[Proof of theorem \ref{theorem:nonintersect-grdpg}]
Define $E_n(\eta)$ as in lemma \ref{lemma:no-noise} and each $\epsilon_i = \|\hat{x}_i - Q_n x_i\|$ where $\hat{x}_i$ is the $i^{th}$ embedding vector and $Q_n$ is some indefinite orthogonal transformation as in \citet{https://doi.org/10.1111/rssb.12509}. 
Let $\epsilon = \max_i \epsilon_i$. 
Then, by theorem 3 of \citet{https://doi.org/10.1111/rssb.12509}, for some finite $M > 0$, $P \big(\epsilon < M \frac{\log^c n}{\sqrt{n}} \big) \to 1$ as $n \to \infty$.
Define $D_n$ as the event that $\epsilon < \delta / 2$. 
For any $\delta > 0$, there is a finite $N$ such that when $n \geq N$, $M \frac{\log^c n}{\sqrt{n}} < \delta / 4$. 
Then for the same $n$, $P(D_n) = 1$, so $\lim\limits_{n \to \infty} P(D_n) = 1$. 

$A_n(\eta)$ is true if both $E_n(\eta)$ and $D_n$ are true, since $E_n(\eta)$ ensures that the $\eta$-neighborhood graph for each manifold is connected and $D_n$ ensures that the $\eta$-neighborhood graph has no edges between embedding points from different manifolds. 

Then to show that $P(A_n) \to 1$: 

$$
\begin{aligned}
P(A_n) & = P(E_n(\eta) \cap D_n) \\
& = P((E_n^c(\eta) \cup D_n^c)^c) \\
& = 1 - P(E_n^c(\eta) \cup D_n^c) \\
& \geq 1 - P(E_n^c(\eta)) - P(D_n^c) \\ 
& = P(E_n(\eta)) + P(D_n) - 1,
\end{aligned}
$$
which tends toward $1$ as $n \to \infty$ since both $P(E_n(\eta))$ and $P(D_n)$ tend toward $1$ as $n \to \infty$. 
\end{proof}

\section{Details on Fitting Bezier Curves with Noise}

\bibliographystyle{agsm}
\bibliography{bibliography.bib}


\end{document}
